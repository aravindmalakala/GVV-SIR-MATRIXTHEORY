\let\negmpace\undefined
\let\negthickspace\undefined
\documentclass[journal]{IEEEtran}
\usepackage[a5paper, margin=10mm, onecolumn]{geometry}
%\usepackage{lmodern} % Ensure lmodern is loaded for pdflatex
\usepackage{tfrupee} % Include tfrupee package
\setlength{\headheight}{1cm} % Set the height of the header box
\setlength{\headsep}{0mm}     % Set the distance between the header box and the top of the text
\usepackage{xparse}
\usepackage{romannum}
\usepackage{gvv-book}
\usepackage{gvv}
\usepackage{cite}
\usepackage{amsmath,amssymb,amsfonts,amsthm}
\usepackage{algorithmic}
\usepackage{graphicx}
\usepackage{textcomp}
\usepackage{xcolor}
\usepackage{txfonts}
\usepackage{listings}
\usepackage{enumitem}
\usepackage{mathtools}
\usepackage{gensymb}
\usepackage{comment}
\usepackage[breaklinks=true]{hyperref}
\usepackage{tkz-euclide} 
\usepackage{listings}
% \usepackage{gvv}                                        
\def\inputGnumericTable{}                                 
\usepackage[latin1]{inputenc}                                
\usepackage{color}                                            
\usepackage{array}                                            
\usepackage{longtable}                                       
\usepackage{calc}                                             
\usepackage{multirow}                                         
\usepackage{hhline}                                           
\usepackage{ifthen}                                           
\usepackage{lscape}
\renewcommand{\thefigure}{\theenumi}
\renewcommand{\thetable}{\theenumi}
\setlength{\intextsep}{10pt} % Space between text and floats
\numberwithin{equation}{enumi}
\numberwithin{figure}{enumi}
\renewcommand{\thetable}{\theenumi}
\begin{document}
\bibliographystyle{IEEEtran}
\title{2021-August Session-01-09-2021-shift-2-21-30}
\author{EE24BTECH11038 - MALAKALA BALA SUBRAHMANYA ARAVIND}
% \maketitle
% \newpage
% \bigskip
{\let\newpage\relax\maketitle}
\begin{enumerate}[start=21]
\item Let X be a random variable with distribution.
\begin{table}[h!]    
  \centering
  \begin{tabular}{|c|c|c|c|c|}
\hline
x & -1 & 0 & 1 & 2\\
\hline
y & 2 & 1 & 2 & 7 \\
\hline
\end{tabular}

  \caption{Variables Used}
\end{table}\\
If the mean of X is 2.3 and variance of X is $\sigma^2$, then 100$\sigma^2$ is equal to :
\bigskip
\item Let $f\brak{x}=x^6+2x^4+x^3+2x+3$, $x\in \mathbf{R}$. Then the value of natural number n such that  
\begin{align*}
    \lim_{x\to 1}\frac{x^nf\brak{1}-f\brak{x}}{x-1}=44
\end{align*}

\bigskip
\item If for the complex numbers z satisfying $\abs{z-2-2i}\leq 1$, the maximum value of $\abs{3iz+6}$ is attained at a+ib, then the value of a+b is equal to 
\bigskip
\item Let the points of intersections of the lines x-y+1=0,x-2y+3 = 0 and 2x-5y+11=0 are the midpoints of the sides of a triangle ABC. Then the area of triangle ABC is
\bigskip
\item Let f$\brak{x}$ be a polynomial of degree 3 such that$f\brak{k}=-\frac{2}{k}$ for k=2,3,4,5. Then the value of 52-10$f\brak{10}$ is equal to :
\bigskip
\item All the arrangements, with or without meaning, of the word FARMER are written excluding any word that has two R appearing together. The arrangements are listed serially in the alphabetic order as in the English dictionary. Then the serial number of the word FARMER in this list is
\bigskip
\item If the sum of the coefficients in the expansion of $\brak{x+y}^n$ is 4096 then the greatest coefficient in the expansion is 
\bigskip
\item Let $\vec{a}= 2\hat{i}-\hat{j} +2\hat{k}$ and $\vec{b}=\hat{i}+ 2\hat{j}-\hat{k}$. Let a vector $\vec{v}$ be in the plane containing $\vec{a} \,\, and\,\, \vec{b}$. If $\vec{v}$ is perpendicular to the vector $3\hat{i}+2\hat{j}-\hat{k}$ and it's projection on $\vec{a}$ is 19 units, then the value of $\abs{2\vec{v}}^2$ is  
\bigskip
\item  Let $\sbrak{t}$ denote the greatest integer $\leq$t. The number of points where the function 
\begin{align*}
    f\brak{x}=\sbrak{x}\abs{x^2-1}+\sin{\brak{\frac{\pi}{\sbrak{x}+3}}}-\sbrak{x+1}, x\in\brak{-2,2}
\end{align*}
is not continuous is.
\bigskip
\item A man starts walking from the point $\vec{P}\brak{-3,4}$ touches the x-axis at R, and then turns to reach at the point $\vec{Q}\brak{0,2}$. The man is walking at a constant speed. If the man reaches the point Q in the minimum time, then $50\brak{\brak{PR}^2+\brak{RQ}^2}$ is equal to 

    \end{enumerate}
\end{document}
