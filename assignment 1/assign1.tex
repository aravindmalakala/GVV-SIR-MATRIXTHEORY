\let\negmedspace\undefined
\let\negthickspace\undefined
\documentclass[journal]{IEEEtran}
\usepackage[a5paper, margin=10mm, onecolumn]{geometry}

%\usepackage{lmodern} % Ensure lmodern is loaded for pdflatex
\usepackage{tfrupee} % Include tfrupee package
\setlength{\headheight}{1cm} % Set the height of the header box
\setlength{\headsep}{0mm}     % Set the distance between the header box and the top of the text
\usepackage{gvv-book}
\usepackage{gvv}
\usepackage{cite}
\usepackage{amsmath,amssymb,amsfonts,amsthm}
\usepackage{algorithmic}
\usepackage{graphicx}
\usepackage{textcomp}
\usepackage{xcolor}
\usepackage{txfonts}
\usepackage{listings}
\usepackage{enumitem}
\usepackage{mathtools}
\usepackage{gensymb}
\usepackage{comment}
\usepackage[breaklinks=true]{hyperref}
\usepackage{tkz-euclide} 
\usepackage{listings}
% \usepackage{gvv}                                        
\def\inputGnumericTable{}                                 
\usepackage[latin1]{inputenc}                                
\usepackage{color}                                            
\usepackage{array}                                            
\usepackage{longtable}                                       
\usepackage{calc}                                             
\usepackage{multirow}                                         
\usepackage{hhline}                                           
\usepackage{ifthen}                                           
\usepackage{lscape}
\begin{document}
\bibliographystyle{IEEEtran}
\vspace{3cm}
\title{FUNCTIONS}
\author{EE24BTECH11038 - MALAKALA BALA SUBRAHMANYA ARAVIND}
% \maketitle
% \newpage
% \bigskip
{\let\newpage\relax\maketitle}
\renewcommand{\thefigure}{\theenumi}
\renewcommand{\thetable}{\theenumi}
\setlength{\intextsep}{10pt} % Space between text and floats
\numberwithin{equation}{enumi}
\numberwithin{figure}{enumi}
\renewcommand{\thetable}{\theenumi}
\section{section b}
\begin{enumerate}[start=19]
\item If the fractional part of the number $\frac{2^{403}}{15}$ is $\frac{k}{15}$, then k is equal to:
    \hfill {(JEE M 2019-9 Jan(M))} 
\begin{enumerate}
    \item $6$
    \item $8$
    \item $4$
    \item $14$
\end{enumerate} 
\item If the function f:$\mathbb{R}-\cbrak{-1,1}$A is defined by f(x)=$\frac{x^2}{1-x^2}$, is surjective then A is equal to:
   \hfill {( JEE M 2019-9 Jan(M))}
\begin{enumerate}
    \item $\mathbb{R}-\{1\}$
    \item $\brak{0,\infty}$
    \item $\mathbb{R}-\lsbrak{-1},\rbrak{0}$
    \item $\brak{-1,0}$
\end{enumerate}
  \item let $\sum\limits_{k=1}^{10}$$f\brak{a+k}$=$16\brak{2^{10}-1}$,where the function f satisfies $f\brak{x+y}=f\brak{x}f\brak{y}$ for all natural numbers x,y and $f\brak{a}=2$.then the natural number `a' is:
    \hfill {( JEE M 2019-9 April(M))} 
\begin{enumerate}
    \item $2$
    \item $16$
    \item $4$
    \item $3$   
\end{enumerate}    
\end{enumerate}
\end{document}
