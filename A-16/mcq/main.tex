\let\negmpace\undefined
\let\negthickspace\undefined
\documentclass[journal]{IEEEtran}
\usepackage[a5paper, margin=10mm, onecolumn]{geometry}
%\usepackage{lmodern} % Ensure lmodern is loaded for pdflatex
\usepackage{tfrupee} % Include tfrupee package
\setlength{\headheight}{1cm} % Set the height of the header box
\setlength{\headsep}{0mm}     % Set the distance between the header box and the top of the text
\usepackage{xparse}
\usepackage{romannum}
\usepackage{gvv-book}
\usepackage{gvv}
\usepackage{cite}
\usepackage{amsmath,amssymb,amsfonts,amsthm}
\usepackage{algorithmic}
\usepackage{graphicx}
\usepackage{textcomp}
\usepackage{xcolor}
\usepackage{txfonts}
\usepackage{listings}
\usepackage{enumitem}
\usepackage{mathtools}
\usepackage{gensymb}
\usepackage{comment}
\usepackage[breaklinks=true]{hyperref}
\usepackage{tkz-euclide} 
\usepackage{listings}
% \usepackage{gvv}                                        
\def\inputGnumericTable{}                                 
\usepackage[latin1]{inputenc}                                
\usepackage{color}                                            
\usepackage{array}                                            
\usepackage{longtable}                                       
\usepackage{calc}                                             
\usepackage{multirow}                                         
\usepackage{hhline}                                           
\usepackage{ifthen}                                           
\usepackage{lscape}
\renewcommand{\thefigure}{\theenumi}
\renewcommand{\thetable}{\theenumi}
\setlength{\intextsep}{10pt} % Space between text and floats
\numberwithin{equation}{enumi}
\numberwithin{figure}{enumi}
\renewcommand{\thetable}{\theenumi}
\begin{document}
\bibliographystyle{IEEEtran}
\title{2024-January Session-31-01-2024-shift-1-16-30}
\author{EE24BTECH11038 - MALAKALA BALA SUBRAHMANYA ARAVIND}
% \maketitle
% \newpage
% \bigskip
{\let\newpage\relax\maketitle}
\begin{enumerate}[start=16]
\item Two marbels are drawn in succession from a box containing 10 red, 30 white, 20 blue and 15 orange marbles, with replacement being made after each drawing. Then the probability, that first drawn marble is red and second drawn marble is white, is
\begin{enumerate}
    \item $\frac{2}{25}$
    \item $\frac{4}{25}$
    \item $\frac{2}{3}$
    \item $\frac{4}{75}$
\end{enumerate}
\bigskip
\item Let $g\brak{x}$ is a linear function and
\begin{align*}
    f\brak{x}=
    \begin{cases}
        g\brak{x}, \,\,\, &x\leq0\\
        \brak{\frac{1+x}{2+x}}^\frac{1}{x} &x>0
    \end{cases}
\end{align*}
is continuous at  x=0 then if $f'\brak{1}=f\brak{-1}$ then the value of $g\brak{3}$ is 
\begin{enumerate}
    \item $\frac{1}{3}\log_{e}\frac{4}{9e^{\frac{1}{3}}}$\\
    \item $\frac{1}{3}\log_{e}\brak{\frac{4}{9}}+1$\\
    \item $\log_{e}\brak{\frac{4}{9}}-1$\\
    \item $\log_{e}\frac{4}{9e^{\frac{1}{3}}}$
\end{enumerate}
\bigskip
\item if $f\brak{x}=\begin{vmatrix}
x^3 & 2x^2 + 1 & 1 + 3x \\
3x^2 + 2 & 2x & x^3 + 6 \\
x^3 - x & 4 & x^2 - 2
\end{vmatrix}$ 
for all $x\in \mathbf{R}$, then $2f\brak{0}+f'\brak{0}$ is equal to 
\begin{enumerate}
    \item 48
    \item 24
    \item 42
    \item 18
\end{enumerate}
\bigskip
\item Three rotten apples are accidently mixed with fifteen good apples. Assuming the random variable x to be the number of rotten apples in a draw of two apples, the variance of x is  
\begin{enumerate}
    \item $\frac{37}{153}$\\
    \item $\frac{57}{153}$\\
    \item $\frac{47}{153}$\\
    \item $\frac{40}{153}$
\end{enumerate}
\bigskip
\item Let S be the set of positive integral values of a for which $\frac{ax^2+2\brak{a+1}x+9a+4}{x^2-8x+32}<0,\forall x \in\mathbf{R}$ then the number of elements in  S
\begin{enumerate}
    \item 1
    \item 0
    \item $\infty$
    \item 3
\end{enumerate}
    \end{enumerate}

\end{document}
