\let\negmpace\undefined
\let\negthickspace\undefined
\documentclass[journal]{IEEEtran}
\usepackage[a5paper, margin=10mm, onecolumn]{geometry}
%\usepackage{lmodern} % Ensure lmodern is loaded for pdflatex
\usepackage{tfrupee} % Include tfrupee package
\setlength{\headheight}{1cm} % Set the height of the header box
\setlength{\headsep}{0mm}     % Set the distance between the header box and the top of the text
\usepackage{xparse}
\usepackage{romannum}
\usepackage{gvv-book}
\usepackage{gvv}
\usepackage{cite}
\usepackage{amsmath,amssymb,amsfonts,amsthm}
\usepackage{algorithmic}
\usepackage{graphicx}
\usepackage{textcomp}
\usepackage{xcolor}
\usepackage{txfonts}
\usepackage{listings}
\usepackage{enumitem}
\usepackage{mathtools}
\usepackage{gensymb}
\usepackage{comment}
\usepackage[breaklinks=true]{hyperref}
\usepackage{tkz-euclide} 
\usepackage{listings}
% \usepackage{gvv}                                        
\def\inputGnumericTable{}                                 
\usepackage[latin1]{inputenc}                                
\usepackage{color}                                            
\usepackage{array}                                            
\usepackage{longtable}                                       
\usepackage{calc}                                             
\usepackage{multirow}                                         
\usepackage{hhline}                                           
\usepackage{ifthen}                                           
\usepackage{lscape}
\renewcommand{\thefigure}{\theenumi}
\renewcommand{\thetable}{\theenumi}
\setlength{\intextsep}{10pt} % Space between text and floats
\numberwithin{equation}{enumi}
\numberwithin{figure}{enumi}
\renewcommand{\thetable}{\theenumi}
\begin{document}
\bibliographystyle{IEEEtran}
\title{2023-April Session-15-04-2023-shift-1-16-30}
\author{EE24BTECH11038 - MALAKALA BALA SUBRAHMANYA ARAVIND}
% \maketitle
% \newpage
% \bigskip
{\let\newpage\relax\maketitle}
\begin{enumerate}[start=21]
\item Let A = $\cbrak{ 1 , 2 , 3, 4}$ and  be a relation on the set A$\times$A defined by R=$\cbrak{\brak{a,b},\brak{c,d}:2a + 3b = 4c + 5d}$. Then
the number of elements in R 
\bigskip
\item The number of elements in the set $\cbrak{n\in N :10\leq n\leq 100 \,\,and\,\, 3^n-3 \,\,is\,\, a \,\,multiple \,\,of\,\, 7}$ is
\bigskip
\item Let an ellipse with centre $\brak{1,0}$ and latus rectum of length $\frac{1}{2}$ have its major axis along x-axis. If its minor axis
subtends an angle 60$\degree$ at the foci, then the square of the sum of the lengths of its minor and major axes is equal
\bigskip
\item If the area bounded by the curve $2y^2=3x$, lines x+y=3,y=0  and outside the circle $\brak{x-3}^2+y^{2}=2$ is A then $4\brak{\pi+4A}$
\bigskip
\item Consider the triangles with vertices $\vec{A}\brak{2,1}, \vec{B}\brak{0,0}$ and $\vec{C}\brak{t, 4}$, t $\in \sbrak{0,4}$ . If the maximum and the minimum perimeters of such triangles are obtained at $t=\alpha$ and $t=\beta$ respectively . Then 6$\alpha$+2$\beta$ is equal to
\bigskip
\item Let the plane P contain the line 2x+y-z-3 = 0 = 5x-3y+4z+9 and be parallel to the line $\frac{x+2}{2}=\frac{3-y}{4}=\frac{z-7}{5}$Then the distance of the point $\vec{A}\brak{8,-1,-19}$ from the plane P measured parallel to the line $\frac{x}{-3}=\frac{y-5}{4}=\frac{2-z}{-12}$ is equal to 
\bigskip
\item If the sum of series $\brak{\frac{1}{2}-\frac{1}{3}}+\brak{\frac{1}{2^2}-\frac{1}{2.3}+\frac{1}{3^2}}+\brak{\frac{1}{2^3}-\frac{1}{2^2.3}+\frac{1}{2.3^2}-\frac{1}{3^3}}+\cdots=\frac{\alpha}{\beta}$. where $\alpha,\beta$ are co-prime then the value of $\alpha+3\beta$
\bigskip
\item A person forgets his 4-digit ATM pin code. But he remembers that in the code all the digits are different, the greatest digit is 7 and the sum of the first two digits is equal to the sum of the last two digits. Then the maximum number of trials necessary to obtain the correct code is
\bigskip
\item  If the line x = y = z intersects the line x$\sin{A} + y\sin{B} + z\sin{C}-18 = 0 = x\sin{2A}+ y\sin{2B}+z\sin{2C}-9$,
where A, B, C are the angles of a triangle ABC, then 80$\brak{\sin{\frac{A}{2}}}\sin{\frac{B}{2}}\sin{\frac{C}{2}}$
\bigskip
\item $f\brak{x}=\int\frac{dx}{\brak{3+4x^2}\sqrt{4-3x^2}},\abs{x}<\frac{2}{\sqrt{3}}$. If $f\brak{0}=0$ and $f\brak{1}=\frac{1}{\alpha \beta}\tan^{-1}\frac{\alpha}{\beta}$ then $\alpha^{2}+\beta^{2}$ is equal to 
    \end{enumerate}

\end{document}
