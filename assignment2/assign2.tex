\let\negmedspace\undefined
\let\negthickspace\undefined
\documentclass[journal]{IEEEtran}
\usepackage[a5paper, margin=10mm, onecolumn]{geometry}
%\usepackage{lmodern} % Ensure lmodern is loaded for pdflatex
\usepackage{tfrupee} % Include tfrupee package
\setlength{\headheight}{1cm} % Set the height of the header box
\setlength{\headsep}{0mm}     % Set the distance between the header box and the top of the text
\usepackage{gvv-book}
\usepackage{gvv}
\usepackage{cite}
\usepackage{amsmath,amssymb,amsfonts,amsthm}
\usepackage{algorithmic}
\usepackage{graphicx}
\usepackage{textcomp}
\usepackage{xcolor}
\usepackage{txfonts}
\usepackage{listings}
\usepackage{enumitem}
\usepackage{mathtools}
\usepackage{gensymb}
\usepackage{comment}
\usepackage[breaklinks=true]{hyperref}
\usepackage{tkz-euclide} 
\usepackage{listings}
% \usepackage{gvv}                                        
\def\inputGnumericTable{}                                 
\usepackage[latin1]{inputenc}                                
\usepackage{color}                                            
\usepackage{array}                                            
\usepackage{longtable}                                       
\usepackage{calc}                                             
\usepackage{multirow}                                         
\usepackage{hhline}                                           
\usepackage{ifthen}                                           
\usepackage{lscape}
\renewcommand{\thefigure}{\theenumi}
\renewcommand{\thetable}{\theenumi}
\setlength{\intextsep}{10pt} % Space between text and floats
\numberwithin{equation}{enumi}
\numberwithin{figure}{enumi}
\renewcommand{\thetable}{\theenumi}
\begin{document}
\bibliographystyle{IEEEtran}
\title{Question-1-1.5-17}
\author{EE24BTECH11038 - MALAKALA BALA SUBRAHMANYA ARAVIND}
% \maketitle
% \newpage
% \bigskip
{\let\newpage\relax\maketitle}
\textbf{Question}:\\
The midpoint of line segment joining $\vec{A}\brak{2a,4}$ and$\vec{B}\brak{-2,3b}$ is $\brak{1,2a+1}$. Find the values of a and b.\\
\textbf{Solution}:\\
\begin{table}[h!]
   \centering
   \begin{tabular}{|c|c|c|c|c|}
\hline
$Y_{i}$ & 3 & -2.5 & 5 & -5\\
\hline
$X_{i}$ & 1 & -2 & 3 & -2 \\
\hline
\end{tabular}

   \caption{variables used}
   \label{tabQuestion-1-1.5-17}
\end{table}   
\begin{align}
    \vec{C}=\frac{\vec{A}+\vec{B}}{2}
\end{align}
Now substituting the values of $\vec{A}$, $\vec{B}$ and $\vec{C}$ 
\begin{align}
    \myvec {1\\2a+1}=\frac{\myvec{2a\\4}+\myvec{-2\\3b}}{2}\\
    \myvec{2\\4a+2}=\myvec{2a\\4} + \myvec{-2\\3b}\\
    \myvec{2\\4a+2}=\myvec{2a-2\\4+3b}
\end{align}
Comparing the matrices we get,
\begin{align}
    2a-2=2\\
    2a=4\\
    a=2\\
\end{align}
The other equation is 
\begin{align}
    4a+2=4+3b\\
    4(2)+2=4+3b\\
    6=3b\\
    b=2
\end{align}
\begin{figure}[h!]
    \centering
    \includegraphics[width=0.5\linewidth]{figs/Figure_1.png}
    \caption{stem plot of line $\vec{A}\vec{B}$}
    \label{stemplot}
\end{figure}
\end{document}
