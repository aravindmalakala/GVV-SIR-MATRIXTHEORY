\let\negmpace\undefined
\let\negthickspace\undefined
\documentclass[journal]{IEEEtran}
\usepackage[a5paper, margin=10mm, onecolumn]{geometry}
%\usepackage{lmodern} % Ensure lmodern is loaded for pdflatex
\usepackage{tfrupee} % Include tfrupee package
\setlength{\headheight}{1cm} % Set the height of the header box
\setlength{\headsep}{0mm}     % Set the distance between the header box and the top of the text
\usepackage{xparse}
\usepackage{romannum}
\usepackage{gvv-book}
\usepackage{gvv}
\usepackage{cite}
\usepackage{amsmath,amssymb,amsfonts,amsthm}
\usepackage{algorithmic}
\usepackage{graphicx}
\usepackage{textcomp}
\usepackage{xcolor}
\usepackage{txfonts}
\usepackage{listings}
\usepackage{enumitem}
\usepackage{mathtools}
\usepackage{gensymb}
\usepackage{comment}
\usepackage[breaklinks=true]{hyperref}
\usepackage{tkz-euclide} 
\usepackage{listings}
% \usepackage{gvv}                                        
\def\inputGnumericTable{}                                 
\usepackage[latin1]{inputenc}                                
\usepackage{color}   
\usepackage{amsmath}
\usepackage{array}                                            
\usepackage{longtable}                                       
\usepackage{calc}                                             
\usepackage{multirow}                                         
\usepackage{hhline}                                           
\usepackage{ifthen}                                           
\usepackage{lscape}
\renewcommand{\thefigure}{\theenumi}
\renewcommand{\thetable}{\theenumi}
\setlength{\intextsep}{10pt} % Space between text and floats
\numberwithin{equation}{enumi}
\numberwithin{figure}{enumi}
\renewcommand{\thetable}{\theenumi}
\begin{document}
\bibliographystyle{IEEEtran}
\title{2019-ST-53 to 65}
\author{EE24BTECH11038 - MALAKALA BALA SUBRAHMANYA ARAVIND}
% \maketitle
% \newpage
% \bigskip
{\let\newpage\relax\maketitle}
\begin{enumerate}[start=53]
\item Let $\mathbf{X}$ be a random variable with uniform distribution on the interval $\brak{-1,1}$ and $\mathbf{X}$ = $\brak{\mathbf{X}+1}^2$. Then the probability density function f$\brak{y}$ of $\mathbf{Y}$, over the interval $\brak{0,4}$, is
\begin{enumerate}
    \item $\frac{3\sqrt{y}}{16}$
    \item $\frac{1}{4\sqrt{y}}$
    \item $\frac{1}{6\sqrt{y}}$
    \item $\frac{1}{\sqrt{y}}$
\end{enumerate}
\bigskip
\item Let $\mathbf{S}$ be the solid whose base is the region in the XY-plane bounded by the curves y = $x^2$ and y = 8-$x^2$, and whose cross-sections perpendicular to the x-axis are squares. Then the volume of $\mathbf{S}$  is
\bigskip
\item Consider the trinomial distribution with the probability mass function $P\brak{X=x,Y=y} = \frac{7!}{x!y!\brak{7-x-y}!}\brak{0.6}^x\brak{0.2}^{y}\brak{0.2}^{7-x-y},x\geq0,y\geq0,\,\, \text{and} x+y\leq 7$ Then $E\brak{Y|X=3}$ is equal to 
\bigskip
\item Let $Y_{i}$ = $\alpha+\beta x_{i}+\in_{i}$, where i = 1, 2, 3, 4, $x_{i}$`s are fixed covariates and $\in$`s are independent and identically distributed standard normal random variables. Here, $\alpha$ and $\beta$ are unknown parameters. Let $\phi$ be the cumulative distribution function of the standard normal distribution and $\phi \brak{1.96}$ = 0.975. Given the following observations,\\
\begin{table}[!ht]
  \centering
  \begin{tabular}{|c|c|c|c|c|}
\hline
$Y_{i}$ & 3 & -2.5 & 5 & -5\\
\hline
$X_{i}$ & 1 & -2 & 3 & -2 \\
\hline
\end{tabular}

  \caption{variables used}
\end{table}\\
the length $\brak{\text{rounded off to two decimal places of the shortest}}$ 95\% confidence interval for $\beta$ based on its least squares estimator is equal to
\bigskip
\item Consider a discrete time Markov chain on the state space $\cbrak{1,2,3}$ with one-step transition  probability matrix $\myvec{0&0.2&0.8\\0.5&0&0.5\\0.6&0.4&0}$
Then the period of the Markov chain is 
\bigskip
\item Suppose customers arrive at an ATM facility according to a Poisson process with rate 5 customers per hour. The probability $\brak{\text{rounded off to two decimal places}}$ that no customer arrives at the ATM facility from 1:00 pm to 1:18 pm is
\bigskip
\item Let $\mathbf{X}$ be a random variable with characteristic function $\phi_{x}\brak{.}$ such that $\phi_{x}\brak{2\pi}$ = 1. Let $\mathbf{Z}$ denote the set of integers. Then P$\brak {X \in \mathbf{Z}}$ is equal to
\bigskip
\item Let $X_1$ be a random sample of size 1 from uniform distribution over $\brak{\theta,\theta^2}$, where $\theta > $1. To test $\mathbf{H}_0$:$\theta$ = 2 against $\mathbf{H}_1$: $\theta$ = 3, reject $\mathbf{H}_0$ if and only if $X_1 > $3.5 . Let $\alpha$ and $\beta$ be the size and the power, respectively, of this test. Then $\alpha$ + $\beta$  is equal to
\bigskip
\item Let $Y_{i}$ = $\beta_0$ + $\beta_1x_i + \in_{i}$, i = 1,$\cdots$,n where $x_{i}$`s are fixed covariates, and $\in `s$ are uncorrelated random variables with mean zero and constant variance. Suppose that $\hat{\beta}_0$ and  $\hat{\beta}_1$ are the least squares estimators of the unknown parameters $\beta_0$ and $\beta_1$, respectively. If $\sum_{i=1}^{n}x_{i}$ = 0 , then the correlation between $\hat{\beta}_0$ and $\hat{\beta}_1$ is equal to
\bigskip
\item Let f:$\mathbf{R}\to\mathbf{R}$ be defined as f$\brak{x}=\brak{3x^2+4}\cos{x}$then $\lim_{h\to 0}\frac{f\brak{h}+f\brak{-h}-8}{h^2}$ is equal to
\bigskip
\item The maximum value of $\brak{x-1}^2$ + $\brak{y-2}^2$ subject to the constraint $x^2$ + $y^2 \leq 45$ is equal to
\bigskip
\item Let $x_1,\cdots x_{10}$ be independent and identically distributed normal random variables with mean 0 and variance 2. Then E$\brak{\frac{x_1^{2}}{x_1^{2}+\cdots x_{10}^2}}$ is equal to
\bigskip
\item Let $\mathbf{I}$ be the 4 $\times$ 4 identity matrix and v = $\brak{1, 2, 3, 4}^t$, where t denotes the transpose. Then the determinant of $\mathbf{I}$ + $vv^t$ is equal to
\end{enumerate}
\end{document}i
