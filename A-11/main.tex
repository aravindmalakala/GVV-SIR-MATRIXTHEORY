\let\negmpace\undefined
\let\negthickspace\undefined
\documentclass[journal]{IEEEtran}
\usepackage[a5paper, margin=10mm, onecolumn]{geometry}
%\usepackage{lmodern} % Ensure lmodern is loaded for pdflatex
\usepackage{tfrupee} % Include tfrupee package
\setlength{\headheight}{1cm} % Set the height of the header box
\setlength{\headsep}{0mm}     % Set the distance between the header box and the top of the text
\usepackage{xparse}
\usepackage{gvv-book}
\usepackage{gvv}
\usepackage{cite}
\usepackage{amsmath,amssymb,amsfonts,amsthm}
\usepackage{algorithmic}
\usepackage{graphicx}
\usepackage{textcomp}
\usepackage{xcolor}
\usepackage{txfonts}
\usepackage{listings}
\usepackage{enumitem}
\usepackage{mathtools}
\usepackage{gensymb}
\usepackage{comment}
\usepackage[breaklinks=true]{hyperref}
\usepackage{tkz-euclide} 
\usepackage{listings}
% \usepackage{gvv}                                        
\def\inputGnumericTable{}                                 
\usepackage[latin1]{inputenc}                                
\usepackage{color}                                            
\usepackage{array}                                            
\usepackage{longtable}                                       
\usepackage{calc}                                             
\usepackage{multirow}                                         
\usepackage{hhline}                                           
\usepackage{ifthen}                                           
\usepackage{lscape}
\renewcommand{\thefigure}{\theenumi}
\renewcommand{\thetable}{\theenumi}
\setlength{\intextsep}{10pt} % Space between text and floats
\numberwithin{equation}{enumi}
\numberwithin{figure}{enumi}
\renewcommand{\thetable}{\theenumi}
\begin{document}
\bibliographystyle{IEEEtran}
\title{2020-September Session-09-02-2020-shift-2-16-25}
\author{EE24BTECH11038 - MALAKALA BALA SUBRAHMANYA ARAVIND}
% \maketitle
% \newpage
% \bigskip
{\let\newpage\relax\maketitle}
\begin{enumerate}[start=16]
   \item Let n be an integer. Suppose that there are n metro stations in a city located along a circular path. Each pair of stations is connected by a straight track only. Further, each pair of nearest stations is connected by blue line, whereas all remaining pairs of stations are connected by red line. If the number of red lines is 99 times the number of blue lines, then the value of n is:
   \begin{enumerate}
       \item 201
       \item 199
       \item 101
       \item 200
   \end{enumerate}
   \item If a curve y = $f\brak{x}$, passing through the point $\brak{1,2}$ is the solution of the differential equation, $2x^2dy = (2xy+y^2)dx$, then $f\brak{\frac{1}{2}}$ is equal to:
   \begin{enumerate}
       \item $-\frac{1}{1+\log_{e}^{2}}$
       \item $\frac{1}{1+\log_{e}^{2}}$
       \item ${1+\log_{e}^{2}}$
       \item $\frac{1}{1-\log_{e}^{2}}$
   \end{enumerate}
   \item  For some $\theta \in \brak{0,\frac{\pi}{2}}$ , if the eccentricity of the hyperbola, $x^2-y^2\sec ^2 \theta = 10$ is $\sqrt{5}$ times the eccentricity of the ellipse, $x^2 \sec^2 \theta+ y^2 = 5$, then the length of the latus rectum of the ellipse, is:
   \begin{enumerate}
       \item $\frac{4\sqrt{5}}{3}$
       \item $\frac{2\sqrt{5}}{3}$
       \item $2\sqrt{6}$
       \item $\sqrt{30}$
   \end{enumerate}
   \item  $lim_{x\rightarrow 0} \brak{\tan\brak{\frac{\pi}{4}+x}}$ is equal to :
   \begin{enumerate}
       \item e
       \item $e^2$
       \item 2
       \item 1
   \end{enumerate}
   \item Let a,b,c $\in \mathbf{R}$ be all non-zero and satisfy $a^3+b^3+c^3=2$. If the matrix A=$\myvec {a & b & c \\ b & c & a \\ c & a & b}$ satisfies $A^TA=I$, then the value of abc can be
   \begin{enumerate}
       \item $\frac{2}{3}$
       \item 3
       \item $-\frac{1}{3}$
       \item $\frac{1}{3}$
   \end{enumerate}
   \item Let the position vectors of points `$\vec{A}$' and `$\vec{B}$' be $\hat{i} +\hat{j}+\hat{k}$ and $2\hat{i} +\hat{j}+3\hat{k}$ respectively. A point `$\vec{P}$' divides the line segment AB internally in the ratio $\lambda$:1 $\brak{\lambda > 0}$ . If $\vec{O}$ is the origin and $\vec{OB}$.$\vec{OP}$ -3$\abs{\vec{OA} X \vec{OP}}^2$=6 then $\lambda$ is :
   \item Let $\sbrak{t}$ denote the greatest integer less than or equal to t. Then the value of
   $\int_{1}^{2}\abs{2x-3\sbrak{3x}}dx$
   \item if Y=$\sum_{k=1}^{6} k\cos^{-1}\cbrak{\frac{3}{5}\cos kx-\frac{4}{5}\sin kx}$ then $\brak{\frac{dy}{dx}}$ at x=0 is 
   \item If the variance of the terms in an increasing A.P., $b_1, b_2, b_3,...., b_{11}$ is 90, then the common difference of this A.P. is
   \item For a positive integer n, $\brak{1+\frac{1}{x}}^n$ is expanded in increasing powers of x. If three consecutive coefficients in this expansion are in the ratio, 2:5:12, then n is equal to

   
   \end{enumerate}
\end{document}
