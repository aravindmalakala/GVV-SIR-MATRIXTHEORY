\let\negmpace\undefined
\let\negthickspace\undefined
\documentclass[journal]{IEEEtran}
\usepackage[a5paper, margin=10mm, onecolumn]{geometry}
%\usepackage{lmodern} % Ensure lmodern is loaded for pdflatex
\usepackage{tfrupee} % Include tfrupee package
\setlength{\headheight}{1cm} % Set the height of the header box
\setlength{\headsep}{0mm}     % Set the distance between the header box and the top of the text
\usepackage{xparse}
\usepackage{gvv-book}
\usepackage{gvv}
\usepackage{cite}
\usepackage{amsmath,amssymb,amsfonts,amsthm}
\usepackage{algorithmic}
\usepackage{graphicx}
\usepackage{textcomp}
\usepackage{xcolor}
\usepackage{txfonts}
\usepackage{listings}
\usepackage{enumitem}
\usepackage{mathtools}
\usepackage{gensymb}
\usepackage{comment}
\usepackage[breaklinks=true]{hyperref}
\usepackage{tkz-euclide} 
\usepackage{listings}
% \usepackage{gvv}                                        
\def\inputGnumericTable{}                                 
\usepackage[latin1]{inputenc}                                
\usepackage{color}                                            
\usepackage{array}                                            
\usepackage{longtable}                                       
\usepackage{calc}                                             
\usepackage{multirow}                                         
\usepackage{hhline}                                           
\usepackage{ifthen}                                           
\usepackage{lscape}
\renewcommand{\thefigure}{\theenumi}
\renewcommand{\thetable}{\theenumi}
\setlength{\intextsep}{10pt} % Space between text and floats
\numberwithin{equation}{enumi}
\numberwithin{figure}{enumi}
\renewcommand{\thetable}{\theenumi}
\begin{document}
\bibliographystyle{IEEEtran}
\title{Question-9-7.2-19}
\author{EE24BTECH11038 - MALAKALA BALA SUBRAHMANYA ARAVIND}
% \maketitle
% \newpage
% \bigskip
{\let\newpage\relax\maketitle}
\textbf{Question}:\\
Find the area of the smaller part of the circle $x^2+y^2=a^2$ cut off by the line $x=\frac{a}{\sqrt{2}}$
\\
\solution \\
\begin{table}[!ht]
  \centering
  \begin{tabular}{|c|c|c|c|c|}
\hline
$Y_{i}$ & 3 & -2.5 & 5 & -5\\
\hline
$X_{i}$ & 1 & -2 & 3 & -2 \\
\hline
\end{tabular}

  \caption{variables used}
  \label{tabQuestion-7-7.2-26}
\end{table}\\
The given circle can be expressed as a conic with parameters
\begin{align}
    \mathbf{V}=\myvec{1 & 0\\0 & 1},\mathbf{u}=0,f=-a^2
\end{align}
Line parameters are 
\begin{align}
    \mathbf{h}=\myvec{\frac{a}{\sqrt{2}} \\ 0}, \mathbf{m}=\mathbf{e}_3
\end{align}
Let the point of intersections with the line to the conic be $\vec{A}$ and $\vec{B}$ 
\begin{align}
    \vec{A}=\myvec{\frac{a}{\sqrt{2}} \\ -\frac{a}{\sqrt{2}}},\vec{B}=\myvec{\frac{a}{\sqrt{2}}  \frac{a}{\sqrt{2}}}
\end{align}
From the Fig.0.1 the area of the portion is given by
\begin{align}
    ar(APQ)&=2ar(APR)\\
    &=2\int_{0}^{\frac{a}{\sqrt{2}}} \left(\sqrt{a^2-x^2} \right) \, dx\\
    &=\frac{a^2}{2}\left(1+\frac{\pi}{2} \right)
\end{align}
\begin{figure}[!ht]
    \centering
    \includegraphics[width=\linewidth]{figs/Figure_2.png}
    \caption{}
\end{figure}

\end{document}
