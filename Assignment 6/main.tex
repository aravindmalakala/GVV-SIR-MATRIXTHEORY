\let\negmpace\undefined
\let\negthickspace\undefined
\documentclass[journal]{IEEEtran}
\usepackage[a5paper, margin=10mm, onecolumn]{geometry}
%\usepackage{lmodern} % Ensure lmodern is loaded for pdflatex
\usepackage{tfrupee} % Include tfrupee package
\setlength{\headheight}{1cm} % Set the height of the header box
\setlength{\headsep}{0mm}     % Set the distance between the header box and the top of the text
\usepackage{xparse}
\usepackage{gvv-book}
\usepackage{gvv}
\usepackage{cite}
\usepackage{amsmath,amssymb,amsfonts,amsthm}
\usepackage{algorithmic}
\usepackage{graphicx}
\usepackage{textcomp}
\usepackage{xcolor}
\usepackage{txfonts}
\usepackage{listings}
\usepackage{enumitem}
\usepackage{mathtools}
\usepackage{gensymb}
\usepackage{comment}
\usepackage[breaklinks=true]{hyperref}
\usepackage{tkz-euclide} 
\usepackage{listings}
% \usepackage{gvv}                                        
\def\inputGnumericTable{}                                 
\usepackage[latin1]{inputenc}                                
\usepackage{color}                                            
\usepackage{array}                                            
\usepackage{longtable}                                       
\usepackage{calc}                                             
\usepackage{multirow}                                         
\usepackage{hhline}                                           
\usepackage{ifthen}                                           
\usepackage{lscape}
\renewcommand{\thefigure}{\theenumi}
\renewcommand{\thetable}{\theenumi}
\setlength{\intextsep}{10pt} % Space between text and floats
\numberwithin{equation}{enumi}
\numberwithin{figure}{enumi}
\renewcommand{\thetable}{\theenumi}
\begin{document}
\bibliographystyle{IEEEtran}
\title{Question-3-3.2-22}
\author{EE24BTECH11038 - MALAKALA BALA SUBRAHMANYA ARAVIND}
% \maketitle
% \newpage
% \bigskip
{\let\newpage\relax\maketitle}
\textbf{Question}:\\
Write true or false in each of the following. Give reasons for your answer\\
1. A triangle can be constructed in wich $\vec{AB}=5cm$, $\angle A=45\degree$ and $\vec{BC}+\vec{AC}=5cm$
\\
\solution \\
\begin{table}[h!]
   \centering
   \begin{tabular}{|c|c|c|c|c|}
\hline
$Y_{i}$ & 3 & -2.5 & 5 & -5\\
\hline
$X_{i}$ & 1 & -2 & 3 & -2 \\
\hline
\end{tabular}

   \caption{variables used}
   \label{tabQuestion-3-3.2-22}
\end{table}\\
By triangle inequality
\begin{align}
    \vec{CB}+\vec{AC} >  \vec{AB} \\
    5cm>5cm
\end{align}
clearly the above statement is False\\
A triangle can't be constructed\\
\\
2. A triangle can be constructed in which $\vec{BC}=6cm$, $\angle B=30\degree$ and $\vec{AC}-\vec{AB}=4cm$.
\\
\solution \\
\begin{table}[h!]
   \centering
   \begin{tabular}{|c|c|c|c|c|}
\hline
x & -1 & 0 & 1 & 2\\
\hline
y & 2 & 1 & 2 & 7 \\
\hline
\end{tabular}

   \caption{variables used}
   \label{tabQuestion-3-3.2-22}
\end{table}\\
Let $\vec{AB}=a$ and $\vec{AC}=a+4$\\
Checking triangle inequalities\\
\begin{align}
    \vec{AB}+\vec{BC} >\vec{AC}\\
    a+6>a+4\\
    \implies True\\
    \vec{AC}+\vec{BC}>\vec{AB}\\
    a+4+6>a\\
    10>0\\
    \implies True\\
    \vec{AB}+\vec{AC}>\vec{BC}\\
    a+a+4>6\\
    a>1
\end{align}
if $\vec{AB}>1$ a triangle can be constructed\\
\begin{figure}[h!]
    \centering
    \includegraphics[width=0.5\linewidth]{figs/Figure_1.png}
    \caption{ Line $\vec{A}\vec{B}$}
    \label{stemplot}
\end{figure}\\

3. A triangle can be constructed in which $\angle B = 105\degree$, $\angle C = 90\degree$ and $\vec{AB}+\vec{BC}+\vec{AC} =10cm$
\\
\solution \\
\begin{table}[h!]
   \centering
   \begin{tabular}{ |c| c|}
    \hline
    \textbf{Given}  & \textbf{Values}\\
    \hline
    $\vec{BC}+\vec{AB}+\vec{AC}$ & $10cm$ \\
    \hline
    $\angle B$ & $105\degree$\\
    \hline
    $\angle C$ & $90\degree$\\
    \hline
\end{tabular}

   \caption{variables used}
   \label{tabQuestion-3-3.2-22}
\end{table}\\
In a triangle the sum of all interior angles should be equal to 180
\begin{align}
    \angle B + \angle C=195
\end{align}
Therefore a triangle cannot be constructed
\end{document}

