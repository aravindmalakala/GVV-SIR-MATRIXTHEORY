\let\negmpace\undefined
\let\negthickspace\undefined
\documentclass[journal]{IEEEtran}
\usepackage[a5paper, margin=10mm, onecolumn]{geometry}
%\usepackage{lmodern} % Ensure lmodern is loaded for pdflatex
\usepackage{tfrupee} % Include tfrupee package
\setlength{\headheight}{1cm} % Set the height of the header box
\setlength{\headsep}{0mm}     % Set the distance between the header box and the top of the text
\usepackage{xparse}
\usepackage{gvv-book}
\usepackage{gvv}
\usepackage{cite}
\usepackage{amsmath,amssymb,amsfonts,amsthm}
\usepackage{algorithmic}
\usepackage{graphicx}
\usepackage{textcomp}
\usepackage{xcolor}
\usepackage{txfonts}
\usepackage{listings}
\usepackage{enumitem}
\usepackage{mathtools}
\usepackage{gensymb}
\usepackage{comment}
\usepackage[breaklinks=true]{hyperref}
\usepackage{tkz-euclide} 
\usepackage{listings}
% \usepackage{gvv}                                        
\def\inputGnumericTable{}                                 
\usepackage[latin1]{inputenc}                                
\usepackage{color}                                            
\usepackage{array}                                            
\usepackage{longtable}                                       
\usepackage{calc}                                             
\usepackage{multirow}                                         
\usepackage{hhline}                                           
\usepackage{ifthen}                                           
\usepackage{lscape}
\renewcommand{\thefigure}{\theenumi}
\renewcommand{\thetable}{\theenumi}
\setlength{\intextsep}{10pt} % Space between text and floats
\numberwithin{equation}{enumi}
\numberwithin{figure}{enumi}
\renewcommand{\thetable}{\theenumi}
\begin{document}
\bibliographystyle{IEEEtran}
\title{Question-3-3.2-22}
\author{EE24BTECH11038 - MALAKALA BALA SUBRAHMANYA ARAVIND}
% \maketitle
% \newpage
% \bigskip
{\let\newpage\relax\maketitle}
\textbf{Question}:\\
Find the direction and normal vectors of the following line:\\
x+2y=6
\\
\solution \\
\begin{table}[!ht]
  \centering
  \begin{tabular}{|c|c|c|c|c|}
\hline
$Y_{i}$ & 3 & -2.5 & 5 & -5\\
\hline
$X_{i}$ & 1 & -2 & 3 & -2 \\
\hline
\end{tabular}

  \caption{variables used}
  \label{tabQuestion-4-4.2-22}
\end{table}\\
\begin{align}
    y &= mx + c \\
    x &= 0 \Rightarrow y = c \\
    x &= 1 \Rightarrow y = mx + c \\
    x &= h + m  
\end{align}
\begin{align}
    m^\top n = 0\\
    n^\top x = n^\top h + \kappa n^\top m \\
    \Rightarrow n (x - h) = 0 \\
    n^\top x = c \\
    c = n^\top h 
\end{align}
\begin{align}
    \text{where} \quad n =\begin{pmatrix}-m \\1\end{pmatrix}
\end{align}
For the line \( 2y + x =  6 \):
\begin{align}
m &= -\frac{1}{2}
\end{align}
\begin{align}
\text{Direction vector } m = \begin{pmatrix} 1 \\ m \end{pmatrix} = \begin{pmatrix} 1 \\ -\frac{1}{2} \end{pmatrix} \\
\implies \begin{pmatrix} -2 \\ 1 \end{pmatrix}
\end{align}
The normal vector is defined by:
\begin{align}
n = \begin{pmatrix} -m \\ 1 \end{pmatrix} = \begin{pmatrix} \frac{1}{2} \\ 1 \end{pmatrix}\\
\implies \begin{pmatrix} 1 \\ 2 \end{pmatrix}
\end{align}
\begin{figure}[!ht]
    \centering
    \includegraphics[width=\linewidth]{figs/Desktop.png}
    \caption{ Line $Line x+2y=6$}
\end{figure}
\end{document}

