\let\negmpace\undefined
\let\negthickspace\undefined
\documentclass[journal]{IEEEtran}
\usepackage[a5paper, margin=10mm, onecolumn]{geometry}
%\usepackage{lmodern} % Ensure lmodern is loaded for pdflatex
\usepackage{tfrupee} % Include tfrupee package
\setlength{\headheight}{1cm} % Set the height of the header box
\setlength{\headsep}{0mm}     % Set the distance between the header box and the top of the text
\usepackage{xparse}
\usepackage{romannum}
\usepackage{gvv-book}
\usepackage{gvv}
\usepackage{cite}
\usepackage{amsmath,amssymb,amsfonts,amsthm}
\usepackage{algorithmic}
\usepackage{graphicx}
\usepackage{textcomp}
\usepackage{xcolor}
\usepackage{txfonts}
\usepackage{listings}
\usepackage{enumitem}
\usepackage{mathtools}
\usepackage{gensymb}
\usepackage{comment}
\usepackage[breaklinks=true]{hyperref}
\usepackage{tkz-euclide} 
\usepackage{listings}
% \usepackage{gvv}                                        
\def\inputGnumericTable{}                                 
\usepackage[latin1]{inputenc}                                
\usepackage{color}                                            
\usepackage{array}                                            
\usepackage{longtable}                                       
\usepackage{calc}                                             
\usepackage{multirow}                                         
\usepackage{hhline}                                           
\usepackage{ifthen}                                           
\usepackage{lscape}
\renewcommand{\thefigure}{\theenumi}
\renewcommand{\thetable}{\theenumi}
\setlength{\intextsep}{10pt} % Space between text and floats
\numberwithin{equation}{enumi}
\numberwithin{figure}{enumi}
\renewcommand{\thetable}{\theenumi}
\begin{document}
\bibliographystyle{IEEEtran}
\title{2021-July Session-20-07-2021-shift-2-21-25}
\author{EE24BTECH11038 - MALAKALA BALA SUBRAHMANYA ARAVIND}
% \maketitle
% \newpage
% \bigskip
{\let\newpage\relax\maketitle}
\begin{enumerate}[start=21]
\item Consider a triangle having vertices $\vec{A}\brak{-2,3},\,\vec{B}\brak{1,9},\,\vec{C}\brak{3,8}$. if a line L passing through the circumcentre of triangle ABC, bisects the line BC, and intersects the Y-axis at $\brak{0,\frac{\alpha}{2}}$, then the value of real number $\alpha$ is
\item Let $\cbrak{a_n}_{n=1}^{\infty}$ be a sequence such that $a_1=1,a_2=1\, and a_{n+2}=2a_{n+1}+1$ for all $n\geq 1$. Then the value of $47\sum_{n=1}^{\infty} \frac{a_n}{2^{3n}}$
\item The number of solutions of the equations $\log_{\brak{x+1}}^{\brak{2x^2+7x+5}}+\log_{\brak{2x+5}}^{\brak{x+1}^2}-4=0$, $x>0$, is 
\item If $\lim_{x\to 0} \frac{\alpha xe^x-\beta\log_{e}^{1+x}+\gamma x^2e^{-x}}{x\sin^2{x}}=10$ ,$\alpha,\,\beta,\,\gamma \in \mathbf{R}$, then the value of $\alpha+\beta+\gamma$ is 
\item For $p>0$, a vector $\vec{v}_2=2\hat{i}+\brak{p+1}\hat{j}$ is obtained by rotating the vector $\vec{v}_1=\sqrt{3}p\hat{i}+\hat{j}$ by an angle $\theta$ about the origin in a counter clock wise direction if $\tan{\theta}=\frac{\alpha \sqrt{3}-2}{4\sqrt{3}+3}$, then the value of $\alpha$ is 
   
   \end{enumerate}
\end{document}
