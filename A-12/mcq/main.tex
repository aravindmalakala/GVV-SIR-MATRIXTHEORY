\let\negmpace\undefined
\let\negthickspace\undefined
\documentclass[journal]{IEEEtran}
\usepackage[a5paper, margin=10mm, onecolumn]{geometry}
%\usepackage{lmodern} % Ensure lmodern is loaded for pdflatex
\usepackage{tfrupee} % Include tfrupee package
\setlength{\headheight}{1cm} % Set the height of the header box
\setlength{\headsep}{0mm}     % Set the distance between the header box and the top of the text
\usepackage{xparse}
\usepackage{romannum}
\usepackage{gvv-book}
\usepackage{gvv}
\usepackage{cite}
\usepackage{amsmath,amssymb,amsfonts,amsthm}
\usepackage{algorithmic}
\usepackage{graphicx}
\usepackage{textcomp}
\usepackage{xcolor}
\usepackage{txfonts}
\usepackage{listings}
\usepackage{enumitem}
\usepackage{mathtools}
\usepackage{gensymb}
\usepackage{comment}
\usepackage[breaklinks=true]{hyperref}
\usepackage{tkz-euclide} 
\usepackage{listings}
% \usepackage{gvv}                                        
\def\inputGnumericTable{}                                 
\usepackage[latin1]{inputenc}                                
\usepackage{color}                                            
\usepackage{array}                                            
\usepackage{longtable}                                       
\usepackage{calc}                                             
\usepackage{multirow}                                         
\usepackage{hhline}                                           
\usepackage{ifthen}                                           
\usepackage{lscape}
\renewcommand{\thefigure}{\theenumi}
\renewcommand{\thetable}{\theenumi}
\setlength{\intextsep}{10pt} % Space between text and floats
\numberwithin{equation}{enumi}
\numberwithin{figure}{enumi}
\renewcommand{\thetable}{\theenumi}
\begin{document}
\bibliographystyle{IEEEtran}
\title{2021-July Session-20-07-2021-shift-2-16-20}
\author{EE24BTECH11038 - MALAKALA BALA SUBRAHMANYA ARAVIND}
% \maketitle
% \newpage
% \bigskip
{\let\newpage\relax\maketitle}
\begin{enumerate}[start=16]
\item  The value of $\tan \brak{{2\tan^{-1}{\frac{3}{5}+\sin^{-1}{\frac{5}{13}}}}}$ is:
\begin{enumerate}
    \item $\frac{220}{21}$\\
    \item $\frac{110}{21}$\\
    \item $\frac{55}{21}$\\
    \item $\frac{20}{11}$
\end{enumerate}
\item  If $\brak{\alpha,\beta}$ is the point on $y^2 = 6x$, that is closest to $\brak{3,\frac{3}{2}}$ then find $2\brak{\alpha+\beta}$
\begin{enumerate}
    \item 6
    \item 9
    \item 7
    \item 5
\end{enumerate}
\item Two circles pass through $\brak{-1,4}$ and their centres lie on $x^2+y^2+2x+4y = 4$. If $r_1$ and $r_2$ are maximum and minimum radii and $\frac{r_1}{r_2}$ = a+$\sqrt{2}$b, then the value of a+b is
\begin{enumerate}
    \item 3
    \item 11
    \item 5
    \item 7
\end{enumerate}
\item If $\triangle ABC$ is a right-angled triangle with sides a,b and c and smallest angle $\theta$. If $\frac{1}{a} , \,\frac{1}{b}\, and \frac{1}{c}$ are also the sides of the right-angle triangle then find $\sin{\theta}$.
\begin{enumerate}
    \item $\sqrt{\frac{\brak{3-\sqrt{5}}}{2}}$\\
    \item $\frac{\brak{3-\sqrt{5}}}{2}$\\
    \item $\sqrt{\frac{\brak{3+\sqrt{5}}}{2}}$\\
    \item ${\frac{\brak{3+\sqrt{5}}}{2}}$
\end{enumerate}
\item For the natural numbers m,n if $\brak{1-y}^m\brak{1+y}^n$=1+$a_1y+a_2y^2+a_3y^3+\cdots a_{m+n}y^{m+n}$ and $a_1=a_2=10,$ then the value of $\brak{m+n}$ is equal to
\begin{enumerate}
    \item 88
    \item 64
    \item 100
    \item 80
\end{enumerate}
\end{enumerate}
\end{document}
